\documentclass{article}



% IDIOMA -----------------------------------------------------------------------
\usepackage[spanish, mexico]{babel}

% GRÁFICOS ---------------------------------------------------------------------
\usepackage{geometry}

% MACROS -----------------------------------------------------------------------
\usepackage{packages/qx-tools}

% FUENTE -----------------------------------------------------------------------
\usepackage{newtxtext, newtxmath} % Se puede cambiar por la fuente con la que se sienta más a gusto
\usepackage[T1]{fontenc}



% OTROS ------------------------------------------------------------------------
\title{Teoría de conjuntos -- Ejercicios}
\author{Qx}
\date{\today}





% Modo oscuro :p
% \definecolor{DarkBg}{RGB}{39, 40, 34}
% \definecolor{DarkTx}{RGB}{248, 248, 242}
% \colorlet{qx-box-back}{blue!5!white!10!DarkBg}

% \pagecolor{DarkBg}
% \color{DarkTx}




\begin{document}

\maketitle



\begin{exercise}[1.1]
  Sea $(a,b) \coloneqq \{\{a\}, \{a,b\}\}$, entonces verifica que $(a,b) = (c,d) \iff a=c \land b=d$.
\end{exercise}
\begin{proof}
  La vuelta es trivial por las propiedades de la igualdad. De este modo, basta con probar la ida.

  Para ello, consideremos que $(a,b) = (c,d)$, entonces, procedamos por casos. En primer lugar, supongamos que $a = b$, entonces $(a,b) = \{\{a\}\}$, sin embargo, esto implica que $\{c\} = \{a\}$ y por ende $a = c$ y de igual forma $d \in \{c,d\} 
  = \{a\}$, por ende $a = b = c = d$.

  Ahora, supongamos que $a \neq b$. Entonces, notemos que $\{a,b\} \neq \{c\}$ ya que en caso contrario $a = c = b$, lo que contradice la hipótesis, de este modo tendríamos que $\{a\} = \{c\}$ y $\{a,b\} = \{c,d\}$. La primera igualdad nos dice que $a = c$, mientras que la segunda igualdad, dado que $b \neq c$, nos dice que $b = d$.
\end{proof}

\begin{exercise}[1.2]
  No existe conjunto $X$ tal que $P(X) \subseteq X$.  
\end{exercise}
\begin{proof}
  Sea $X$ un conjunto cualquiera, usando el axioma esquema de separación, consideremos el siguiente subconjunto de $X$
  \[
    S = \{ u \in X : u \notin u \}.
  \]
  Ahora, si $S \in X$ entonces tendríamos que $S \in S$ o $S \notin S$. Sin embargo, si $S \in S$, entonces por construcción, tendríamos que $S \notin S$ y de forma análoga, si $S \notin S$, entonces eso implicaría que $S \in S$, en ambos casos tenemos una contradicción. De este modo $S \notin X$ y por ende $P(X) \nsubseteq X$.
\end{proof}

Sea $\N = \bigcap\{X : X \ \text{es inductivo}\}$. $\N$ es el conjunto inductivo mas pequeño. Consideremos la siguiente notación
\[
  0 = \emptyset,\quad
  1 = \{0\},\quad
  2 = \{0,1\},\quad
  3 = \{0,1,2\},\quad
  \ldots
\]
Si $n \in \N$, definamos $n+1 = n \cup \{n\}$. Definamos $<$ (en $\N$) dado por $n < m$ si y solo si $n \in m$.

Un conjunto $T$ es \emph{transitivo} si $x \in T$ implica que $x \subseteq T$.

\begin{exercise}[1.3]
  Si $X$ es inductivo, entonces el conjunto $\{x \in X : x \subseteq X\}$ es inductivo. Por lo tanto $\N$ es transitivo y para cada $n$ se tiene que $n = \{m \in \N : m<n\}$.
\end{exercise}
\begin{proof}
  En primer lugar, dado que $X$ es inductivo, entonces tenemos que $\emptyset \in X$ y de igual manera, por construcción tenemos que $\emptyset \subseteq X$, de este modo si $S = \{ x \in X : x \subseteq X\}$, entonces tenemos que $\emptyset \in S$. De igual manera, si $x \in S \subseteq X$, entonces tenemos que $x \cup \{x\} \in X$, dado que $X$ es inductivo y $x \cup \{x\} \subseteq X$ por construcción, de este modo tenemos que $x\cup\{x\} \in S$, lo que finalmente muestra que $S$ es inductivo.

  Ahora, dado que $\N$ es inductivo, entonces tenemos que $S = \{n \in \N : n \subseteq \N\}$ es inductivo, sin embargo, como $S \subseteq \N \subseteq S$, dado que $\N$ es el conjunto inductivo más pequeño, entonces tenemos que $\N = S$, lo que implica que para todo $n \in \N$ se cumple que $n \subseteq \N$, lo que quiere decir que $\N$ es transitivo.

  Finalmente, notemos que por definición $\{n \in \N : m < n\} = n \cap \N$, sin embargo, como $n \subseteq \N$, entonces tenemos que $n = n \cap \N$, lo que finalmente muestra que $n = \{m \in \N: m<n\}$.
\end{proof}

\begin{exercise}[1.4]
  Si $X$ es inductivo, entonces el conjunto $\{ x \in X: x \ \text{es transitivo}\}$ es inductivo. Por lo tanto todo $n \in \N$ es transitivo.
\end{exercise}
\begin{proof}
  Sea $S = \{ x \in X: x \ \text{es transitivo}\}$, dado que $X$ es inductivo, entonces tenemos que $\emptyset \in X$, de este modo basta con probar que $\emptyset$ es transitivo. Ahora, recordemos que la definición de que un conjunto $A$ es inductivo es si $\forall x (x \in A \to x \subseteq A)$, de este modo, tomando $A = \emptyset$ entonces tenemos que 
  \[
    \forall x (x \in \emptyset \lthen x \subseteq \emptyset) \iff \forall x (\lfalse \lthen x \subseteq \emptyset) \iff \ltrue,
  \]
  donde $\top$ denota el símbolo para una tautología y $\bot$ el símbolo para una contradicción. De este modo, tenemos que $\emptyset$ es transitivo y por ende $\emptyset \in S$. Ahora, sea $x \in S \subseteq X$, entonces, por hipótesis, es claro que $x \cup \{x\} \in X$, de este modo basta con probar que $x \cup \{x\}$ es transitivo. Sea $u \in x \cup \{x\}$, si $u \in x$, dado que $x$ es transitivo, entonces tenemos que $u \subseteq x \subseteq x \cup \{x\}$, mientras que si $u \in \{x\}$, entonces tenemos que $u = x$ y por ende $u \subseteq x \cup \{x\}$, lo que finalmente muestra la transitividad y por ende $x \cup \{x\} \in S$. Así tenemos que $S$ es inductivo.

  Ahora, análogamente al ejercicio anterior, tenemos que $S = \{ x \in \N: x \ \text{es transitivo}\}$ es inductivo, sin embargo, como $\N$ es el conjunto inductivo más pequeño, entonces tenemos que $\N = S$, lo que muestra que para todo $n \in \N$ se cumple que $n$ es transitivo.
\end{proof}

\begin{exercise}[1.5]
  Si $X$ es inductivo, entonces el conjunto $\{x \in X : x \ \text{es transitivo y} \ x \notin x\}$ es inductivo. Por lo tanto $n \notin n$ y $n \neq n+1$ para cada $n \in \N$.
\end{exercise}
\begin{proof}
  Sea $S = \{x \in X : x \ \text{es transitivo y} \ x \notin x\}$, por lo mostrado en el ejercicio anterior, tenemos que $\emptyset \in X$ y además es inductivo, además, por construcción tenemos que $\emptyset \notin \emptyset$, de este modo $\emptyset \in S$. De igual manera, tenemos que si $x \in S$, entonces $x \cup \{x\} \in X$ y dado que $x$ es transitivo, entonces $x \cup \{x\}$ es transitivo, así basta con probar que $x \cup \{x\} \notin x \cup \{x\}$. Ahora, si $x \cup \{x\} \in x$, dado que $x$ es transitivo, entonces esto implica que $x \cup \{x\} \subseteq x$ lo que implica que $x \in x$, lo que es una contradicción, mientras que si $x \cup \{x\} \in \{x\}$, entonces esto implica que $x \cup \{x\} = x$, lo que nuevamente implica que $x \in x$. De este modo, tenemos que $x \cup \{x\} \notin x$ y $x \cup \{x\} \notin \{x\}$ y por ende $x \cup \{x\} \notin x \cup \{x\}$. De este modo $x \cup \{x\} \in S$ y por ende $S$ es inductivo.

  De esta forma, análogamente a ejercicios anteriores, tenemos que $S = \{x \in \N : x \ \text{es transitivo y} \ x \notin x\}$ es inductivo y por minimalidad tenemos que $\N = S$. De este modo, para todo $n \in \N$ tenemos que $n \notin n$ y de igual manera, tenemos que $n \neq n+1$, ya que en caso contrario tendríamos que $n = n \cup \{n\}$, lo que implicaría que $n \in n$.
\end{proof}

\begin{exercise}[1.6]
  Si $X$ es inductivo, entonces $\{x \in X : x$ es transitivo y todo $z \subseteq x$ no vacío tiene un elemento $\in$-minimal$\}$ es inductivo ($t$ es \emph{$\in$-minimal} en $z$ si no existe $s \in z$ tal que $s \in t$).
\end{exercise}
\begin{proof}
  Sea $S = \{x \in X : x$ es transitivo y todo $z \subseteq x$ no vacío tiene un elemento $\in$-minimal$\}$, notemos que $\emptyset \in S$ dado que el vacío es transitivo y no tiene subconjuntos no vacíos. Ahora, sea $x \in S$, entonces por lo mostrado en ejercicios anteriores, tenemos que $x \cup \{x\} \in X$ y además es transitivo, así basta ver cualquiera de sus subconjuntos no vacíos tiene un elemento $\in$-minimal. Sea $z \subseteq x \cup \{x\}$ no vacío, entonces procedamos por casos, si $x \notin z$, entonces tenemos que $z \subseteq x$, de este modo, por hipótesis tenemos que tiene un elemento $\in$-minimal, así, basta con mostrar el caso cuando $x \in z$.

  En primer lugar, notemos que $x \notin x$, ya que en caso contrario $\{x\}$ es un subconjunto de $x$ no vacío que no tiene un elemento $\in$-minimal. De este modo, supongamos que $x \in z$, sea $z' = z - \{x\} \subseteq x$, entonces tenemos dos casos, si $z' = \emptyset$, eso quiere decir que $z = \{x\}$ y por ende $x$ es un elemento $\in$-minimal de $z$. Ahora, si $z' \neq \emptyset$, dado que $z' \subseteq x$, entonces, por hipótesis, tenemos que tiene un elemento $t$ que es $\in$-minimal, así si $s \in z'$ entonces por la minimalidad de $t$ tenemos que $s \notin t$, mientras que si $s = x$, entonces tenemos que si $s \notin t$ ya que en caso contrario tendríamos que $x \in t \in x$, sin embargo, por la transitividad de $x$, tendríamos que $t \subseteq x$, lo que implicaría que $x \in x$, lo cual es una contradicción. De este modo $t$ es un elemento $\in$-minimal de $z = z' \cup \{x\}$. Lo que finalmente muestra que $x \cup \{x\} \in S$ y por ende $S$ es transitivo.
\end{proof}

\begin{exercise}[1.7]
  Todo conjunto $X \subseteq \N$ no vacío tiene un elemento $\in$-minimal.
\end{exercise}
\begin{proof}
  Análogamente a lo realizado en ejercicios anteriores, por la minimalidad de $\N$, tenemos que $\N = \{x \in \N : x$ es transitivo y todo $z \subseteq x$ no vacío tiene un elemento $\in$-minimal$\}$.

  Ahora, sea $X \subseteq \N$ no vacío, entonces tenemos que existe $n \in X$, así, consideremos $z = (n+1) \cap X \subseteq n+1$, notemos que es no vacío ya que $n \in z$, de este modo, por lo mostrado al inicio, tenemos que existe $t \in z$ tal que es $\in$-minimal, así, mostremos que la $\in$-minimalidad también se aplica para $X$. Así, sea $s \in X$, si $s \in n+1$, entonces por construcción, tenemos que $s \notin t$, mientras que si $s \notin n+1$, dado que $t \in n+1$, entonces por la transitividad de $n+1$, tenemos que $t \subseteq n+1$, de este modo $s \notin t$ ya que en caso contrario, tendríamos que $s \notin n+1$. De esta forma, tenemos que $t$ es $\in$-minimal en $X$.
\end{proof}

\begin{exercise}[1.8]
  Si $X$ es inductivo, entonces también lo es $\{x \in X : x = \emptyset \ \text{o} \ x = y \cup \{y\} \ \text{para algún $y$}\}$. Por lo tanto cada $n \in \N$ con $n \neq 0$ es $m + 1$ para algún $m \in \N$.
\end{exercise}
\begin{proof}
  Sea $S = \{x \in X : x = \emptyset \ \text{o} \ x = y \cup \{y\} \ \text{para algún $y$}\}$, entonces $\emptyset \in S$ por construcción, de este modo tomemos $x \in S$, por la inductividad de $X$ tenemos que $x \cup\{x\} \in X$ y por construcción $x \cup\{x\} \in S$.

  De esta forma, análogamente a lo realizado en ejercicios anteriores, por la minimalidad de $\N$ tenemos que $\N = \{x \in \N : x = \emptyset \ \text{o} \ x = y \cup \{y\} \ \text{para algún $y$}\}$, de esta forma, si $n \neq 0$, entonces tenemos que existe $m$ tal que $n = m \cup\{m\}$, sin embargo, por el ejercicio 1.3 tenemos que $n = \{k \in \N : k < n\}$, entonces dado que $m \in n$, esto quiere decir que $m \in \N$ y por ende $n = m+1$.
\end{proof}

\begin{exercise}[1.9 (Inducción)]
  Sea $A$ un conjunto de $\N$ tal que $0 \in A$ y si $n \in A$ entonces $n+1 \in A$. Entonces $A = \N$.
\end{exercise}
\begin{proof}
  Supongamos que $A \neq \N$, entonces sea $X = \N - A$, entonces tenemos que $X$ es un subconjunto de $\N$ no vacío, por ende existe $t \in X$ tal que es $\in$-minimal. Ahora, por el ejercicio anterior, tenemos que $t = n+1$, para algún $n \in \N$, dado que $n < t$ entonces esto quiere decir que $n \notin X$, por ende $n \in A$. Sin embargo, por hipótesis tenemos que $t = n+1 \in A$, lo que contradice que $t \in X$. De este modo tenemos que $X = \emptyset$ y por ende $A = \N$.
\end{proof}

Un conjunto $X$ \emph{tiene $n$ elementos} (donde $n \in \N$) si existe un mapeo inyectivo de $n$ en $X$. Un conjunto es \emph{finito} si tiene $n$ elementos para algún $n \in \N$ y es \emph{infinito} si no es finito.

Un conjunto $S$ es \emph{T-finito} si cada $X \subseteq P(S)$ no vacío tiene un elemento $\subseteq$-maximal, i.e., $u \in X$ tal que no existe $v \in X$ con $u \subseteq v$ y $u \neq v$. $S$ es \emph{T-infinito} si no es T-finito.

\begin{exercise}[1.10]
  Cada $n \in \N$ es T-finito.
\end{exercise}


\begin{exercise}[1.11]
  $\N$ es T-infinito; el conjunto $\N \subseteq P(\N)$ no tiene un elemento $\subseteq$-maximal.
\end{exercise}

\begin{exercise}[1.12]
  Cada conjunto finito es T-finito.
\end{exercise}
\begin{proof}
  En primer lugar, notemos que si $f \colon X \to Y$ es una función biyectiva, entonces podemos construir una función $F \colon P(X) \to P(Y)$ biyectiva si consideramos $F(A) = f(A)$. Para ver que es sobre, basta ver que si $B \subseteq Y$ entonces
  \[
    F\bigl(f^{-1}(B)\bigr) = \bigl\{ f^{-1}\bigl(f(b)\bigr) : b \in B \bigr\} = B,
  \]
  donde $f^{-1}\colon Y \to X$ es la función inversa de $f$, mientras que la inyectividad se da ya que si $F(A) = F(B)$ eso quiere decir que para cada $f(a) \in f(A)$ existe $b \in f(B)$ tal que $f(a) = f(b)$ y por ende $a = b$ lo que muestra que $A = B$. Más aun, la función $F$ preserva la inclusión, es decir $A \subseteq B$ si y solo si $F(A) \subseteq F(B)$.

  Con esto probemos la proposición. En primer lugar, sea $X$ un conjunto finito, esto quiere decir que existe un $n \in \N$ y función $f \colon n \to X$ biyectiva y por ende podemos construir una función $F \colon P(n) \to P(X)$ biyectiva que preserva la inclusión.

  De esta forma, sea $S \subseteq P(X)$ un conjunto no vacío, por lo mostrado anteriormente, tenemos que existe $R \subseteq P(n)$ tal que $F(R) = S$, sin embargo, por el ejercicio anterior, tenemos que $R$ contiene un elemento $u$ que es $\subseteq$-maximal, mostremos que $F(u)$ es $\subseteq$-maximal en $S$. Sea $v \in S$ tal que $F(u) \subseteq v$, dado que $F$ preserva inclusiones, tenemos que $u \subseteq F^{-1}(v) $, sin embargo por la $\subseteq$-maximalidad de $u$, esto implica que $u = F^{-1}(v)$ y por ende $F(u) = v$, lo que finalmente muestra que $F(u)$ es $\subseteq$-maximal en $S$ y por ende $X$ es T-finito.
\end{proof}

\begin{exercise}[1.13]
  Cada conjunto infinito es T-infinito.
\end{exercise}
\begin{proof}
  Probemos que todo $n \in \N$ tiene $n$ elementos, si $n \neq 0$ entonces se satisface inmediatamente mediante la función identidad $\Id_n \colon n \to n$ dada por $\Id_n(k) = k$. Ahora, en el caso $n = 0 = \emptyset$ se satisface ya que $\emptyset \subseteq 0 \times \emptyset$ y cumple que 
  \[
    \forall x \forall y \forall z \bigl( (x,y) \in \emptyset \land (x,z) \in \emptyset \lthen y = z \bigr) 
    \iff
    \forall x \forall y \forall z (\lfalse  \lthen y = z )
    \iff \ltrue.
  \]
  De este modo $\emptyset$ es una función $ 0 \to \emptyset$, además, por un razonamiento análogo, tenemos que es biyectiva.

  Ahora, probemos la contrarrecíproca. Sea $S$ un conjunto T-finito, en este caso consideremos el conjunto $X = \{x \in P(S) : x \ \text{es finito}\}$, notemos que $X$ es no vacío ya que $\emptyset \in X$. De esta forma, por hipótesis, tenemos que existe $u \in X$ tal que es $\subseteq$-maximal, si $u = S$ entonces tenemos que $S$ es finito, por lo que basta considerar el caso cuando $u \neq S$.

  Dado que $u$ es finito, entonces tenemos que existe $n \in \N$ y una función $f\colon u \to n$ biyectiva. Ahora, como $u \neq S$, entonces tenemos que existe $s \in S$ tal que $s \notin u$, de esta forma consideremos la función $f'\colon u \cup \{s\} \to n+1$ dado por $f' = f \cup (s, n)$. Por construcción, tenemos que $f'$ es biyectiva, de esta forma $u \cup \{s\} \in X$, sin embargo, esto contradice la $\subseteq$-maximalidad de $u$, de esta forma, este caso no es posible, lo que muestra que $S$ es finito.
\end{proof}

\begin{exercise}[1.14]
  El axioma de separación se sigue del esquema de reemplazo.
\end{exercise}
\begin{proof}
  Sea $\varphi(x,p)$ una formula lógica, consideremos la clase $F$ dada por $F = \{ (x,x) : \varphi(x,p) \}$. Es inmediato que es un funcional ya que si $(x,y), (x,z) \in F$, entonces esto quiere decir que $y = x= z$.

  De esta forma, sean $X$ y $p$ un conjuntos, por el esquema de reemplazo, tenemos que $Y = F(X)$ es un conjunto. Ahora, sea $y \in Y$, por definición, tenemos que existe $x \in X$ tal que $ y = F(x) = x$, sin embargo, como $y \in \dom(F) = \{x : \varphi(x,p)\}$, esto quiere decir que $y \in \{x \in X : \varphi(x,p)\}$ y por ende
  \[
    Y \subseteq \{x \in X : \varphi(x,p)\}.
  \]

  Análogamente, sea $y \in \{x \in X : \varphi(x,p)\}$, entonces tenemos que $y \in X \cap \dom(F)$ lo que implica que $y = F(y) \in Y$. De este modo, tenemos que $Y = \{x \in X : \varphi(x,p)\}$, lo que muestra el axioma de separación.
\end{proof}

\begin{exercise}[1.15]
  En vez de los axiomas de unión, conjunto potencia y reemplazo, considera las siguientes versiones débiles:
  \begin{enumerate}
    \item $\forall X \exists Y (\bigcup X \subseteq Y)$,\qquad i.e., $\forall X \exists Y  \forall x \in X  \forall u \in x  (u \in Y)$.
    
    \item $\forall X  \exists Y  (P(X) \subseteq Y)$,\qquad i.e., $\forall X \exists Y  \forall u  (u \subseteq X \lthen u \in Y)$.
    
    \item Si $F$ es un funcional, entonces $\forall X  \exists Y  (F(X) \subseteq Y)$.
  \end{enumerate}
  Entonces, los axiomas de unión, conjunto potencia y reemplazo, pueden ser probados con 1, 2 y 3, usando el esquema de separación.
\end{exercise}




\end{document}

