Un conjunto $X$ \emph{tiene $n$ elementos} (donde $n \in \N$) si existe un mapeo inyectivo de $n$ en $X$. Un conjunto es \emph{finito} si tiene $n$ elementos para algún $n \in \N$ y es \emph{infinito} si no es finito.

Un conjunto $S$ es \emph{T-finito} si cada $X \subseteq P(S)$ no vacío tiene un elemento $\subseteq$-maximal, i.e., $u \in X$ tal que no existe $v \in X$ con $u \subseteq v$ y $u \neq v$. $S$ es \emph{T-infinito} si no es T-finito.

\begin{exercise}[1.10]
  Cada $n \in \N$ es T-finito.
\end{exercise}