\begin{exercise}[1.8]
  Si $X$ es inductivo, entonces también lo es $\{x \in X : x = \emptyset \ \text{o} \ x = y \cup \{y\} \ \text{para algún $y$}\}$. Por lo tanto cada $n \in \N$ con $n \neq 0$ es $m + 1$ para algún $m \in \N$.
\end{exercise}
\begin{proof}
  Sea $S = \{x \in X : x = \emptyset \ \text{o} \ x = y \cup \{y\} \ \text{para algún $y$}\}$, entonces $\emptyset \in S$ por construcción, de este modo tomemos $x \in S$, por la inductividad de $X$ tenemos que $x \cup\{x\} \in X$ y por construcción $x \cup\{x\} \in S$.

  De esta forma, análogamente a lo realizado en ejercicios anteriores, por la minimalidad de $\N$ tenemos que $\N = \{x \in \N : x = \emptyset \ \text{o} \ x = y \cup \{y\} \ \text{para algún $y$}\}$, de esta forma, si $n \neq 0$, entonces tenemos que existe $m$ tal que $n = m \cup\{m\}$, sin embargo, por el ejercicio 1.3 tenemos que $n = \{k \in \N : k < n\}$, entonces dado que $m \in n$, esto quiere decir que $m \in \N$ y por ende $n = m+1$.
\end{proof}