\begin{exercise}[1.6]
  Si $X$ es inductivo, entonces $\{x \in X : x$ es transitivo y todo $z \subseteq x$ no vacío tiene un elemento $\in$-minimal$\}$ es inductivo ($t$ es \emph{$\in$-minimal} en $z$ si no existe $s \in z$ tal que $s \in t$).
\end{exercise}
\begin{proof}
  Sea $S = \{x \in X : x$ es transitivo y todo $z \subseteq x$ no vacío tiene un elemento $\in$-minimal$\}$, notemos que $\emptyset \in S$ dado que el vacío es transitivo y no tiene subconjuntos no vacíos. Ahora, sea $x \in S$, entonces por lo mostrado en ejercicios anteriores, tenemos que $x \cup \{x\} \in X$ y además es transitivo, así basta ver cualquiera de sus subconjuntos no vacíos tiene un elemento $\in$-minimal. Sea $z \subseteq x \cup \{x\}$ no vacío, entonces procedamos por casos, si $x \notin z$, entonces tenemos que $z \subseteq x$, de este modo, por hipótesis tenemos que tiene un elemento $\in$-minimal, así, basta con mostrar el caso cuando $x \in z$.

  En primer lugar, notemos que $x \notin x$, ya que en caso contrario $\{x\}$ es un subconjunto de $x$ no vacío que no tiene un elemento $\in$-minimal. De este modo, supongamos que $x \in z$, sea $z' = z - \{x\} \subseteq x$, entonces tenemos dos casos, si $z' = \emptyset$, eso quiere decir que $z = \{x\}$ y por ende $x$ es un elemento $\in$-minimal de $z$. Ahora, si $z' \neq \emptyset$, dado que $z' \subseteq x$, entonces, por hipótesis, tenemos que tiene un elemento $t$ que es $\in$-minimal, así si $s \in z'$ entonces por la minimalidad de $t$ tenemos que $s \notin t$, mientras que si $s = x$, entonces tenemos que si $s \notin t$ ya que en caso contrario tendríamos que $x \in t \in x$, sin embargo, por la transitividad de $x$, tendríamos que $t \subseteq x$, lo que implicaría que $x \in x$, lo cual es una contradicción. De este modo $t$ es un elemento $\in$-minimal de $z = z' \cup \{x\}$. Lo que finalmente muestra que $x \cup \{x\} \in S$ y por ende $S$ es transitivo.
\end{proof}