Sea $\N = \bigcap\{X : X \ \text{es inductivo}\}$. $\N$ es el conjunto inductivo mas pequeño. Consideremos la siguiente notación
\[
  0 = \emptyset,\quad
  1 = \{0\},\quad
  2 = \{0,1\},\quad
  3 = \{0,1,2\},\quad
  \ldots
\]
Si $n \in \N$, definamos $n+1 = n \cup \{n\}$. Definamos $<$ (en $\N$) dado por $n < m$ si y solo si $n \in m$.

Un conjunto $T$ es \emph{transitivo} si $x \in T$ implica que $x \subseteq T$.

\begin{exercise}[1.3]
  Si $X$ es inductivo, entonces el conjunto $\{x \in X : x \subseteq X\}$ es inductivo. Por lo tanto $\N$ es transitivo y para cada $n$ se tiene que $n = \{m \in \N : m<n\}$.
\end{exercise}
\begin{proof}
  En primer lugar, dado que $X$ es inductivo, entonces tenemos que $\emptyset \in X$ y de igual manera, por construcción tenemos que $\emptyset \subseteq X$, de este modo si $S = \{ x \in X : x \subseteq X\}$, entonces tenemos que $\emptyset \in S$. De igual manera, si $x \in S \subseteq X$, entonces tenemos que $x \cup \{x\} \in X$, dado que $X$ es inductivo y $x \cup \{x\} \subseteq X$ por construcción, de este modo tenemos que $x\cup\{x\} \in S$, lo que finalmente muestra que $S$ es inductivo.

  Ahora, dado que $\N$ es inductivo, entonces tenemos que $S = \{n \in \N : n \subseteq \N\}$ es inductivo, sin embargo, como $S \subseteq \N \subseteq S$, dado que $\N$ es el conjunto inductivo más pequeño, entonces tenemos que $\N = S$, lo que implica que para todo $n \in \N$ se cumple que $n \subseteq \N$, lo que quiere decir que $\N$ es transitivo.

  Finalmente, notemos que por definición $\{n \in \N : m < n\} = n \cap \N$, sin embargo, como $n \subseteq \N$, entonces tenemos que $n = n \cap \N$, lo que finalmente muestra que $n = \{m \in \N: m<n\}$.
\end{proof}