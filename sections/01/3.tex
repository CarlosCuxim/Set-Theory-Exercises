Sea $\N = \bigcap\{X : X \ \text{es inductivo}\}$. $\N$ es el conjunto inductivo mas pequeño. Consideremos la siguiente notación
\[
  0 = \emptyset,\quad
  1 = \{0\},\quad
  2 = \{0,1\},\quad
  3 = \{0,1,2\},\quad
  \ldots
\]
Si $n \in \N$, definamos $n+1 = n \cup \{n\}$. Definamos $<$ (en $\N$) dado por $n < m$ si y solo si $n \in m$.

Un conjunto $T$ es \emph{transitivo} si $x \in T$ implica que $x \subseteq T$.

\begin{exercise}[1.3]
  Si $X$ es inductivo, entonces el conjunto $\{x \in X : x \subseteq X\}$ es inductivo. Por lo tanto $\N$ es transitivo y para cada $n$ se tiene que $n = \{m \in \N : m<n\}$.
\end{exercise}