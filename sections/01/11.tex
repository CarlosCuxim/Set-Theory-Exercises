
\begin{exercise}[1.11]
  $\N$ es T-infinito; el conjunto $\N \subseteq P(\N)$ no tiene un elemento $\subseteq$-maximal.
\end{exercise}
\begin{proof}
  Dado que para todo $n \in \N$ tenemos que $n \subseteq \N$, entonces es claro que $\N \subseteq P(\N)$. Ahora, sea $n \in \N$, notemos que $n+1 \in \N$ es un elemento tal que $n \subseteq n+1$ y $n \neq n+1$, de esta forma, ningún $n \in \N$ es $\subseteq$-maximal y por ende $\N$ es T-infinito.
\end{proof}
